\section{Probabilistic Analysis of Linkage Tensor}
We first introduce a lemma which may assist us to apply lemma \ref{lemma:expectation_inverse_gaussian} in lemma \ref{lemma:probabilistic_err_linkage_tensor}. 
\begin{lem}
\label{lemma:indepence_for_pseduo_inverse}
Follow the definition of $\mathbf{\Omega}_n$ and $\mathbf{Q}_n$, 
for any $1\le n\le N$, $\mathbf{\Phi}_n \mathbf{Q}_n$ and 
$\mathbf{\Phi}_n (\mathbf{I} - \mathbf{Q}_n\mathbf{Q}_n^\top)$ are conditional independent given $\mathbf{Q}_n$. 
\begin{proof}
Considering both $(\mathbf{\Phi}_n \mathbf{Q}_n)$ and $(\mathbf{\Phi}_n (\mathbf{I} -\mathbf{Q}_n\mathbf{Q}_n^\top))$ are both Gaussian matrix, it suffices to show that covariance matrix between $\mbox{vec}(\mathbf{\Phi}_n \mathbf{Q}_n)$ and $\mbox{vec}(\mathbf{\Phi}_n (\mathbf{I} -\mathbf{Q}_n\mathbf{Q}_n^\top))$ with $\mathbf{Q}_n$ fixed, is the identity. Note: $\mbox{vec}(AB) = (B^\top \otimes I)\mbox{vec}(A)$. 
\begin{equation}
\begin{aligned}
&\cov(\mbox{vec}(\mathbf{\Phi}_n \mathbf{Q}_n), \mbox{vec}(\mathbf{\Phi}_n (\mathbf{I} -\mathbf{Q}_n\mathbf{Q}_n^\top)))  = \left(\mathbf{Q}_n^\top \otimes \mathbf{I}_k\right)\left((\mathbf{I} - \mathbf{Q}\mathbf{Q}_n^\top) \otimes \mathbf{I}_k\right)\\
&=\left(\mathbf{Q}_n^\top(\mathbf{I} - \mathbf{Q}_n\mathbf{Q}_n^\top)  \right)\otimes \left(\mathbf{I}_k\right) = \mathbf{I}_{k^2} 
\end{aligned}
\end{equation}
completes the proof. 
\end{proof}
\end{lem}

\begin{lem}
\label{lemma:probabilistic_err_linkage_tensor}
\begin{equation}
\mathbb{E} \|\mathscr{W} - \mathscr{X}\times_1 \mathbf{Q}_1^\top \times \cdots \times_N \mathbf{Q}_N^\top\|_F^2\le 
\left(1+\frac{k}{s-k}\right)^N  \sum_{n=1}^N \left(1+\frac{k}{k-\rho-1}\right)(\tau^{(n)}_\rho)^2. 
\end{equation}
\begin{proof}
To make notation succinct, we borrow the concept and notation of filtration in probability:
\begin{equation}
\mathcal{F}_n = \sigma(\mathbf{\Omega}_1, \mathbf{\Phi}_1, \dots, \mathbf{\Omega}_n, \mathbf{\Phi}_n), 
\end{equation}
where $\sigma(\mathbf{\Omega}_1, \mathbf{\Phi}_1, \dots, \mathbf{\Omega}_n, \mathbf{\Phi}_n)$ is the sigma algebra generated by $\{\mathbf{\Omega}_i, \mathbf{\Phi}_i, 1\le i\le n\}$. Let $\mathcal{F}_0 = \{\Omega, \emptyset\}$,  the trivial sigma algebra. Here $\Omega$ is the whole space. In usage, it just simplifies writing the conditional probability:
\begin{equation}
\mathbb{E} (X \mid \mathcal{F}_n) = \mathbb{E} (X\mid \mathbf{Q}_1,\mathbf{\Omega}_1, \dots, \mathbf{Q}_n,\mathbf{\Omega}_n). 
\end{equation}

\begin{equation}
\begin{aligned}
&\mathscr{W} = \mathscr{Z}\times_1 (\mathbf{\Phi}_1 \mathbf{Q}_1)^\dag \times \cdots \times_N (\mathbf{\Phi}_N \mathbf{Q}_N)^\dag \\
& = (\mathscr{X} -  \tilde{\mathscr{X}})\times_1 \mathbf{\Phi}_1 \times \cdots \times_N \mathbf{\Phi}_N  \times_1 (\mathbf{\Phi}_1 \mathbf{Q}_1)^\dag \times \cdots \times_N (\mathbf{\Phi}_N \mathbf{Q}_N)^\dag 
\\
&+ \tilde{\mathscr{X}}\times_1 \mathbf{\Phi}_1 \times \cdots \times_N \mathbf{\Phi}_N \times_1 (\mathbf{\Phi}_1 \mathbf{Q}_1)^\dag \times \cdots \times_N (\mathbf{\Phi}_N \mathbf{Q}_N)^\dag.
\end{aligned}
\end{equation}
Noticing 
\begin{equation}
\begin{aligned}
&\tilde{\mathscr{X}}\times_1 \mathbf{\Phi}_1 \times \cdots \times_N \mathbf{\Phi}_N \times_1 (\mathbf{\Phi}_1 \mathbf{Q}_1)^\dag \times \cdots \times_N (\mathbf{\Phi}_N \mathbf{Q}_N)^\dag   \\
& = \mathscr{X}\times_1 (\mathbf{\Phi}_1 \mathbf{Q}_1)^\dag \mathbf{\Phi}_1\mathbf{Q}_1\mathbf{Q}_1^\top \times \cdots \times_N (\mathbf{\Phi}_N \mathbf{Q}_N)^\dag \mathbf{\Phi}_N\mathbf{Q}_N\mathbf{Q}_N^\top\\
& = \mathscr{X}\times_1 \mathbf{Q}_1^\top \times \cdots \times_N \mathbf{Q}_N^\top.
\end{aligned}
\end{equation}
The last equation comes from the fact that for a matrix $\mathbf{A}$ with shape $s\times k$, if $s>k$, $A^\dag A = I_k$. Therefore
\begin{equation}
\begin{aligned}
&\mathscr{W} = (\mathscr{X} -  \tilde{\mathscr{X}})\times_1 \mathbf{\Phi}_1 \times \cdots \times_N \mathbf{\Phi}_N  \times_1 (\mathbf{\Phi}_1 \mathbf{Q}_1)^\dag \times \cdots \times_N (\mathbf{\Phi}_N \mathbf{Q}_N)^\dag \\
&+ \mathscr{X}\times_1 \mathbf{Q}_1^\top \times \cdots \times_N \mathbf{Q}_N^\top.
\end{aligned}
\end{equation}
Now it suffices to bound the first term in above equation. Applying \eqref{eq: tensor_product_mul_exchangable} and \eqref{eq: tensor_product_association}, we could rearrange first part as
\begin{equation}
\begin{aligned}
&(\mathscr{X} -  \tilde{\mathscr{X}})\times_1 \mathbf{\Phi}_1 \times \cdots \times_N \mathbf{\Phi}_N  \times_1 (\mathbf{\Phi}_1 \mathbf{Q}_1)^\dag \times \cdots \times_N (\mathbf{\Phi}_N \mathbf{Q}_N)^\dag \\
& =(\mathscr{X} -  \tilde{\mathscr{X}})\times_1 (\mathbf{\Phi}_1\mathbf{Q}_1)^\dag \mathbf{\Phi}_1 \times \cdots \times_N (\mathbf{\Phi}_N\mathbf{Q}_N)^\dag \mathbf{\Phi}_N.
\end{aligned}
\end{equation}
We use a recursive argument to bound above equation. Let 
$\mathscr{Y}_0 = (\mathscr{X} -  \tilde{\mathscr{X}})$ and for $1\le n \le N$, 
\begin{equation}
\mathscr{Y}_n =  (\mathscr{X} -  \tilde{\mathscr{X}})\times_1 (\mathbf{\Phi}_1\mathbf{Q}_1)^\dag \mathbf{\Phi}_1 \times \cdots \times_n (\mathbf{\Phi}_n\mathbf{Q}_n)^\dag \mathbf{\Phi}_n.
\end{equation}
For $0\le n<N$, 
\begin{equation}
\label{eq:recursion}
\mathbb{E} [\|\mathscr{Y}_{n+1}\|_F^2] = \mathbb{E}\left\{ \mathbb{E} [\|\mathscr{Y}_{n+1}\|_F^2 \mid \mathcal{F}_n] \right\} = \mathbb{E} \left\{\mathbb{E}_{\mathbf{Q}_{n+1}, \mathbf{\Omega}_{n+1}}  \|\mathscr{Y}_n\times_{(n+1)}\right (\mathbf{\Phi}_{n+1}\mathbf{Q}_{n+1})^\dag \mathbf{\Phi}_{n+1}\|_F^2\}. 
\end{equation}
Here, we use the fact that $\mathbf{Q}_{n+1}$ only depends on $\mathbf{\Omega}_{n+1}$. Noticing $\mathscr{Y}_n$, $\mathbf{\Phi}_{n+1}$ and $\mathbf{Q}_{n+1}$ are independent with each other, 

\begin{equation}
\label{eq:conditional_recursion}
\begin{aligned}
&\mathbb{E}_{\mathbf{Q}_{n+1}, \mathbf{\Omega}_{n+1}}  \|\mathscr{Y}_n\times_{(n+1)} (\mathbf{\Phi}_{n+1}\mathbf{Q}_{n+1})^\dag \mathbf{\Phi}_{n+1}\|_F^2\\
& = \mathbb{E}_{\mathbf{Q}_{n+1}, \mathbf{\Omega}_{n+1}} \|(\mathbf{\Phi}_{n+1}\mathbf{Q}_{n+1})^\dag \mathbf{\Phi}_{n+1} \mathbf{Y}_n^{(n+1)}\|_F^2\\
& = \mathbb{E}_{\mathbf{Q}_{n+1}, \mathbf{\Omega}_{n+1}} \|(\mathbf{\Phi}_{n+1}\mathbf{Q}_{n+1})^\dag \mathbf{\Phi}_{n+1}(\mathbf{Q}_{n+1}\mathbf{Q}_{n+1}^\top +\mathbf{I}-\mathbf{Q}_{n+1}\mathbf{Q}_{n+1}^\top ) \mathbf{Y}_n^{(n+1)}\|_F^2\\
& = \mathbb{E}_{\mathbf{Q}_{n+1}, \mathbf{\Omega}_{n+1}} \|(\mathbf{\Phi}_{n+1}\mathbf{Q}_{n+1})^\dag \mathbf{\Phi}_{n+1}(\mathbf{Q}_{n+1}\mathbf{Q}_{n+1}^\top)\mathbf{Y}_n^{(n+1)}\|_F^2\\
&+\mathbb{E}_{\mathbf{Q}_{n+1}, \mathbf{\Omega}_{n+1}} \|(\mathbf{\Phi}_{n+1}\mathbf{Q}_{n+1})^\dag \mathbf{\Phi}_{n+1}(\mathbf{I}-\mathbf{Q}_{n+1}\mathbf{Q}_{n+1}^\top)\mathbf{Y}_n^{(n+1)}\|_F^2\\
&+2\mathbb{E}_{\mathbf{Q}_{n+1}, \mathbf{\Omega}_{n+1}}  \langle(\mathbf{\Phi}_{n+1}\mathbf{Q}_{n+1})^\dag \mathbf{\Phi}_{n+1}(\mathbf{Q}_{n+1}\mathbf{Q}_{n+1}^\top)\mathbf{Y}_n^{(n+1)},  (\mathbf{\Phi}_{n+1}\mathbf{Q}_{n+1})^\dag \mathbf{\Phi}_{n+1}(\mathbf{I}-\mathbf{Q}_{n+1}\mathbf{Q}_{n+1}^\top)\mathbf{Y}_n^{(n+1)}\rangle. 
\end{aligned}
\end{equation}
Since $\mathbf{Q}_{n+1}$ is independent of $\mathbf{\Omega}_{n+1}$, the last line in \ref{eq:conditional_recursion} becomes 
\begin{equation}
\begin{aligned}
&\mathbb{E}_{\mathbf{Q}_{n+1}, \mathbf{\Omega}_{n+1}}  \langle\mathbf{Q}_{n+1}^\top\mathbf{Y}_n^{(n+1)},  (\mathbf{\Phi}_{n+1}\mathbf{Q}_{n+1})^\dag \mathbf{\Phi}_{n+1}(\mathbf{I}-\mathbf{Q}_{n+1}\mathbf{Q}_{n+1}^\top)\mathbf{Y}_n^{(n+1)}\rangle\\
&= \mathbb{E}_{\mathbf{Q}_{n+1}} \langle\mathbf{Q}_{n+1}^\top\mathbf{Y}_n^{(n+1)}, \mathbb{E}_{\mathbf{\Omega}_{n+1}}  (\mathbf{\Phi}_{n+1}\mathbf{Q}_{n+1})^\dag \mathbf{\Phi}_{n+1}(\mathbf{I}-\mathbf{Q}_{n+1}\mathbf{Q}_{n+1}^\top)\mathbf{Y}_n^{(n+1)} \mid \mathbf{Q}_{n+1}\rangle. 
\end{aligned}
\end{equation}



Now lemma \ref{lemma:indepence_for_pseduo_inverse} claims that $\mathbf{\Phi}_n\mathbf{Q}_n$ and $\mathbf{\Phi}_n(\mathbf{I} - \mathbf{Q}_n\mathbf{Q}_n^\top)$ are independent given $\mathbf{Q}_n$. Besides, 
$\mathbb{E} \mathbf{\Phi}_{n+1}(\mathbf{I}-\mathbf{Q}_{n+1}\mathbf{Q}_{n+1}^\top )\mid \mathbf{Q}_{n+1} = 0$, then above equation becomes zero. \par 
The fact that the expectation of inner product is zero leads to another Pythagorean Theorem type decomposition of \ref{eq:conditional_recursion} as
\begin{equation}
\begin{aligned}
&\mathbb{E}_{\mathbf{Q}_{n+1}, \mathbf{\Omega}_{n+1}}  \|\mathscr{Y}_n\times_{(n+1)} (\mathbf{\Phi}_{n+1}\mathbf{Q}_{n+1})^\dag \mathbf{\Phi}_{n+1}\|_F^2\\
& = \mathbb{E}_{\mathbf{Q}_{n+1}, \mathbf{\Omega}_{n+1}} \|(\mathbf{\Phi}_{n+1}\mathbf{Q}_{n+1})^\dag \mathbf{\Phi}_{n+1}(\mathbf{Q}_{n+1}\mathbf{Q}_{n+1}^\top)\mathbf{Y}_n^{(n+1)}\|_F^2\\
&+\mathbb{E}_{\mathbf{Q}_{n+1}, \mathbf{\Omega}_{n+1}} \|(\mathbf{\Phi}_{n+1}\mathbf{Q}_{n+1})^\dag \mathbf{\Phi}_{n+1}(\mathbf{I}-\mathbf{Q}_{n+1}\mathbf{Q}_{n+1}^\top)\mathbf{Y}_n^{(n+1)}\|_F^2.\\
\end{aligned}
\end{equation}
Since $s>k$, $(\mathbf{\Phi}_{n+1}\mathbf{Q}_{n+1})^\dag \mathbf{\Phi}_{n+1}\mathbf{Q}_{n+1} = \mathbf{I}_k$, 
\begin{equation}
\|(\mathbf{\Phi}_{n+1}\mathbf{Q}_{n+1})^\dag \mathbf{\Phi}_{n+1}(\mathbf{Q}_{n+1}\mathbf{Q}_{n+1}^\top)\mathbf{Y}_n^{(n+1)}\|_F^2 = \|\mathbf{Q}_{n+1}^\top\mathbf{Y}_n^{(n+1)}\|_F^2 \le \|\mathbf{Y}_n^{(n+1)}\|_F^2.
\end{equation}
 For the second part, let $\mathbf{B}_{1,n+1} = (\mathbf{\Phi}_{n+1}\mathbf{Q}_{n+1})^\dag$, $\mathbf{B}_{2,n+1} = \mathbf{\Phi}_{n+1}(\mathbf{I}-\mathbf{Q}_{n+1}\mathbf{Q}_{n+1}^\top)$, 
 as we show above, each element in $\mathbf{B}_{1,n+1}$ is independent from each element in $\mathbf{B}_{2,n+1}$. Then 
\begin{equation}
\begin{aligned}
&\mathbb{E}_{\mathbf{Q}_{n+1}, \mathbf{\Omega}_{n+1}} \|(\mathbf{\Phi}_{n+1}\mathbf{Q}_{n+1})^\dag \mathbf{\Phi}_{n+1}(\mathbf{I}-\mathbf{Q}_{n+1}\mathbf{Q}_{n+1}^\top)\mathbf{Y}_n^{(n+1)}\|_F^2\\
& =\mathbb{E}_{\mathbf{Q}_{n+1}} \mathbb{E}_{\mathbf{B}_{1,n+1}, \mathbf{B}_{2, n+1}} \|\mathbf{B}_{1,n+1}^\dag
\mathbf{B}_{2,n+1}\mathbf{Y}_n^{(n+1)}\|_F^2\\
& = \frac{k}{s-k}\|\mathbf{Y}_n^{(n+1)}\|_F^2.
\end{aligned}
\end{equation}
Then we get a recursive relation as 
\begin{equation}
\mathbb{E}\|\mathscr{Y}_{n}\|_F^2 \le  \left(1+\frac{k}{s-k}\right) \mathbb{E}\|\mathscr{Y}_{n-1}\|_F^2, n\ge 1, 
\end{equation}
which leads to 
\begin{equation}
\mathbb{E}\|\mathscr{Y}_{N}\|_F^2 \le  \left(1+\frac{k}{s-k}\right)^N  \mathbb{E}\|\mathscr{Y}_0\|_F^2 \le  \left(1+\frac{k}{s-k}\right)^N  \sum_{n=1}^N \left(1+\frac{k}{k-\rho-1}\right)(\tau^{(n)}_\rho)^2, 
\end{equation}
where the last inequality comes from lemma \ref{lemma: compression_error}. 
\end{proof}
\end{lem}